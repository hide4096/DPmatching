\documentclass[a4paper,12pt]{article}

\usepackage[utf8]{inputenc}
\usepackage{graphicx}
\usepackage{titletoc}
\usepackage{lipsum}
\usepackage{listings}
\usepackage{filecontents}

\usepackage[top=2.5cm, bottom=2.5cm, left=2.5cm, right=2.5cm]{geometry}

\lstset{
  basicstyle={\ttfamily},
  identifierstyle={\small},
  commentstyle={\smallitshape},
  keywordstyle={\small\bfseries},
  ndkeywordstyle={\small},
  stringstyle={\small\ttfamily},
  frame={tb},
  breaklines=true,
  columns=[l]{fullflexible},
  numbers=left,
  xrightmargin=0zw,
  xleftmargin=3zw,
  numberstyle={\scriptsize},
  stepnumber=1,
  numbersep=1zw,
  lineskip=-0.5ex
}

\title{DPマッチングによる単語音声の認識}
\author{麻生英寿 \\ 21C1005}
\date{\today}

\begin{document}

\maketitle

\tableofcontents

\section{目的}

DPマッチングのアルゴリズムを利用して、小語彙の音声認識の実験を行う。

\section{実験方法}

DPマッチングのアルゴリズムによって音声データの単語間距離を計算するプログラムを作成する。
そのプログラムに、100単語の音声データのテンプレートに対して、同じ発声内容の100単語を未知入力音声として、順に入力していく。
入力された音声データの発声内容を判定し、その正答率を計算する。

\subsection{使用したデータセットについて}

この実験で使用した音声データセットには、2人の話者がそれぞれ同じ100種類の単語を発話したものが含まれている。
話者は100種類の単語の発話を2回行っているため、合計400個の音声データが含まれている。
1つの音声データにはファイル名、発声内容、フレーム数とフレーム数分の15次のメルケプストラム特徴量である。

\subsection{単語間距離の計算}

実験ではDPマッチングのアルゴリズムを用いて2つの音声データの単語間距離を計算する。
単語間距離の計算は以下の手順で行う。

入力として与えられた2つの音声データのフレーム長をそれぞれ$I$と$J$とする。
$(i,j),0<i\leq I ,0<j\leq J$で表せられる2つの音声データの各ノードのメルケプストラム特徴量の距離を計算し、
$(i,j)$でのフレームの距離を局所距離$d(i,j)$と表す。
音声データの最初のフレームから任意のフレームまでの局所距離の総和を累積距離$g(i,j)$とする。
最終フレームでの累積距離$g(I,J)$が最小となるような経路を探索することで、単語間距離を計算する。

\subsection{最終フレームでの累積距離の計算}

はじめに、初期条件を以下のように設定する。
\[
    g(0,0) = d(0,0)
\]
次に境界条件を以下のように設定する。
\[
    j>0 \rightarrow g(0,j) = g(0,j-1) + d(0,j)
\]
\[
    i>0 \rightarrow g(i,0) = g(i-1,0) + d(i,0)
\]
最後に、再帰的に以下の式を用いて、最終フレームでの累積距離$g(I,J)$を計算する。
\[
g(i, j)=\min \left[\begin{array}{llr}
g(i, j-1) & + & d(i, j) \\
g(i-1, j-1) & + & 2 d(i, j) \\
g(i-1, j) & + & d(i, j)
\end{array}\right]
\]

\section{実験結果}

音声認識の正答率は次のようになった。

\begin{table}[htbp]
    \centering
    \begin{tabular}{|c|c|c|c|c|}
      \hline
      モデル/認識対象 & city011 & city012 & city021 & city022 \\
      \hline
      city011 &       & 99\% & 90\%  & 84\% \\
      \hline
      city012 & 100\% &      & 92\%  & 86\% \\
      \hline
      city021 & 83\%  & 91\% &       & 99\% \\
      \hline
      city022 & 86\% &  94\% & 100\%  &      \\
      \hline
    \end{tabular}
    \caption{音声認識の正答率}
    \label{tab:音声認識の正答率}
  \end{table}

\newpage

\section{考察}
実験結果より、同一話者よりも異なる話者の音声データの方が正答率が低いことが読み取れる。
\begin{figure}[h]
    \centering
    \includegraphics[bb=0.000000 0.000000 1329.122658 697.681395,width=1.0\hsize]{./equal.png}
    \caption{同じ音声データ}
    \label{fig:equal}
\end{figure}

\begin{figure}[h]
    \centering
    \includegraphics[bb=0.000000 0.000000 1329.122658 697.681395,width=1.0\hsize]{./same.png}
    \caption{同一話者}
    \label{fig:same}
\end{figure}

\begin{figure}[h]
    \centering
    \includegraphics[bb=0.000000 0.000000 1329.122658 697.681395,width=1.0\hsize]{./diffpeople.png}
    \caption{異なる話者}
    \label{fig:diffpeople}
\end{figure}

\begin{figure}[h]
    \centering
    \includegraphics[bb=0.000000 0.000000 1329.122658 697.681395,width=1.0\hsize]{./diffword.png}
    \caption{異なる話者かつ異なる単語}
    \label{fig:diffworde}
\end{figure}

\newpage

\appendix
\section{付録}
この実験で使用したプログラムのソースコードを以下に示す。

\subsection{付録A: 単語間距離を求めるプログラム}
\lstinputlisting[language=c]{../two_words.c}

\subsection{付録B:正答率を計算するプログラム}
\lstinputlisting[language=c]{../benchmark.py}

\subsection{付録C:可視化するプログラム}
\lstinputlisting[language=c]{../heatmap.py}



\end{document}
