\documentclass[a4paper,12pt]{article}

\usepackage[utf8]{inputenc}
\usepackage{graphicx}
\usepackage{titletoc}
\usepackage{lipsum}

\usepackage[top=2.5cm, bottom=2.5cm, left=2.5cm, right=2.5cm]{geometry}

\title{DPマッチングによる単語音声の認識}
\author{麻生英寿 \\ 21C1005}
\date{\today}

\begin{document}

\maketitle

\tableofcontents

\section{目的}
DPマッチングのアルゴリズムを利用して、小語彙の音声認識の実験を行う。

\section{実験方法}
DPマッチングのアルゴリズムによって音声データの単語間距離を計算するプログラムを作成する。
そのプログラムに、100単語の音声データのテンプレートに対して、同じ発声内容の100単語を未知入力音声として、順に入力していく。
入力された音声データの発声内容を判定し、その正答率を計算する。

\subsection{使用したデータセットについて}
この実験で使用した音声データセットには、2人の話者がそれぞれ同じ100種類の単語を発話したものが含まれている。
話者は100種類の単語の発話を2回行っているため、合計400個の音声データが含まれている。
1つの音声データにはファイル名、発声内容、フレーム数とフレーム数分の15次のメルケプストラム特徴量である。

\subsection{単語間距離の計算}
実験ではDPマッチングのアルゴリズムを用いて2つの音声データの単語間距離を計算する。
単語間距離の計算は以下の手順で行う。

入力として与えられた2つの音声データのフレーム長をそれぞれ$I$と$J$とする。
$(i,j),0<i<=I,0<j<=J$で表せられる2つの音声データの各ノードのメルケプストラム特徴量の距離を計算し、
$(i,j)$でのフレームの距離を局所距離$d(i,j)$と表す。
音声データの最初のフレームから任意のフレームまでの局所距離の総和を累積距離$g(i,j)$とする。
最終フレームでの累積距離$g(I,J)$が最小となるような経路を探索することで、単語間距離を計算する。

\subsection{最終フレームでの累積距離の計算}
はじめに、初期条件を以下のように設定する。
\[
    g(0,0) = d(0,0)
\]
次に境界条件を以下のように設定する。
\[
    j>0 \rightarrow g(0,j) = g(0,j-1) + d(0,j)
\]
\[
    i>0 \rightarrow g(i,0) = g(i-1,0) + d(i,0)
\]
最後に、再帰的に以下の式を用いて、最終フレームでの累積距離$g(I,J)$を計算する。
\[
g(i, j)=\min \left[\begin{array}{llr}
g(i, j-1) & + & d(i, j) \\
g(i-1, j-1) & + & 2 d(i, j) \\
g(i-1, j) & + & d(i, j)
\end{array}\right]
\]

\section{実験結果}
音声認識の正答率は次のようになった。

\begin{table}[htbp]
    \centering
    \begin{tabular}{|c|c|c|c|c|}
      \hline
      モデル/認識対象 & city011 & city012 & city021 & city022 \\
      \hline
      city011 &       & 99\% & 90\%  & 84\% \\
      \hline
      city012 & 100\% &      & 92\%  & 86\% \\
      \hline
      city021 & 83\%  & 91\% &       & 99\% \\
      \hline
      city022 & 86\% &  94\% & 100\%  &      \\
      \hline
    \end{tabular}
    \caption{音声認識の正答率}
    \label{tab:音声認識の正答率}
  \end{table}

\section{考察}

\appendix
\section{付録}
付録に関する内容をここに書きます。

\subsection{付録A}
付録Aの内容をここに書きます。

\subsection{付録B}
付録Bの内容をここに書きます。

\newpage

\section*{参考文献}
\addcontentsline{toc}{section}{参考文献}

ここに参考文献の引用を書きます。

\begin{thebibliography}{9}
\bibitem{example_reference1}
  著者名,
  \textit{タイトル},
  出版社,
  年.
  
\bibitem{example_reference2}
  著者名,
  \textit{タイトル},
  出版社,
  年.
\end{thebibliography}

\end{document}
